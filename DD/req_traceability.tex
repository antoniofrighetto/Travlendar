\chapter{Requirements Traceability}
\label{cha:req_trace}

Here we explain how the system's requirements, previously defined in the RASD document, can map the aforementioned design components.

\begin{table}[h]
\begin{center}
\begin{tabular}{|l|m{0.7\textwidth}|}
\hline
{\bf Component (DD)}  & {\bf Requirements (RASD)}\\
\hline
AuthenticationService & \begin{itemize} \item User registration with own form. \item User login through third-party services. \item User profile management. \end{itemize} \\
\hline
SyncUserInfoService & \begin{itemize} \item Permits user events to be synchronized across different platforms. \end{itemize} \\
\hline
EventService & \begin{itemize} \item Appointment schedules organization. \item Provide overview of appointments through a grid and all schedules through a list view. \item Create a new appointment. \item Modify an existent appointment. \item Delete an appointment. \end{itemize} \\
\hline
MapService & \begin{itemize} \item Travel route arrangement (location of sharing systems, find optimal paths, etc.) \end{itemize} \\
\hline
PurchaseService & \begin{itemize} \item Permits user to buy tickets within the application (if possible).
\end{itemize} \\
\hline
\end{tabular}
\caption{Table that maps components to their respective functional requirements.}
\label{tab:comparisontable}
\end{center}
\end{table}